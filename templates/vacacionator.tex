% \VAR{etiqueta}

\encabezado{Resolución de la Dirección Territorial de Educación de Las Palmas, 
por la que se concede vacaciones a \VAR{dondona} \VAR{nombre} ante su próxima 
jubilación.}

\subsection*{ANTECEDENTES DE HECHO}

\noindent Habiéndose concedido la jubilación a \VAR{dondona} \VAR{nombre}, con 
DNI \VAR{dni}, personal docente en la fecha \VAR{fecha}, le corresponde, en 
consecuencia, el disfrute de \VAR{dias} días hábiles de vacaciones.

\subsection*{CONSIDERACIONES JURÍDICAS}
\begin{enumerate}

\item La Ley 2/1987, de 30 de marzo, de la Función Pública Canaria en su 
artículo 45 apartado 2 g), establece que los funcionarios tendrán derecho 
\enquote{A las vacaciones retribuidas en los términos previstos en la 
legislación básica}.

\item El artículo 50 de la Ley 7/2007, de 12 de abril, del Estatuto Básico del 
Empleado Público establece que \enquote{Los funcionarios públicos tendrán 
derecho a disfrutar, durante cada año natural, de unas vacaciones retribuidas de
 veintidós días hábiles, o de los días que correspondan proporcionalmente si el 
 tiempo de servicio durante el año fue menor}.

\item La Resolución de 1 de agosto de 2006, por la que se determina el período 
vacacional y régimen de permisos y licencias del personal docente no 
universitario al servicio de la Administración Educativa en el ámbito de la 
Comunidad Autónoma de Canarias, establece en su apartado tercero que 
\enquote{El personal sujeto al ámbito de aplicación de esta Resolución, tendrá 
derecho al disfrute de un mes de vacaciones anuales retribuidas que, por razones 
del servicio, será necesariamente el mes de agosto, sin que quepa suspenderlas 
para trasladarlas a otro período}.

\item Las Instrucciones de 7 de abril de 2017 de la Dirección General de 
Personal de la Consejería de Educación y Universidades, establece en su primer 
apartado que el ámbito temporal para el cálculo de las vacaciones del personal 
docente que concluye su relación laboral con esta Administración será el curso 
escolar, esto es, desde el 1 de septiembre al 31 de agosto del año siguiente.

\item Asimismo, en su apartado segundo,  establece que 
los días de vacaciones \emph{a efectos de su disfrute} se calculan conforme a la
siguiente fórmula. 

\begin{quotation}

\begin{equation*}
  \frac{t \times 22 }{365} = {d}
\end{equation*}

\espacio

\noindent Donde $t$ es el tiempo de servicio durante el curso y $d$ el número de
 días hábiles de vacaciones.

\end{quotation}

\item Por último, su apartado cuarto establece que:

\begin{enumerate}

    \item con carácter general las vacaciones de los docentes que concluyan su 
    relación laboral antes del 31 de agosto serán concedidas de oficio por las 
    Direcciones Territoriales
    
    \item \emph{Excepcionalmente}, cuando existan razones de indispensable 
    necesidad del alumnado, que deberán ser debidamente motivas por el Director 
    del Centro afectado, éste informará de esta circunstancia a la Dirección 
    Territorial competente, la cual, vista la información dictará Resolución, 
    en su caso. No obstante, dicho informe del Director del Centro afectado será
    expresado por escrito y formará parte del expediente del interesado.
    
\end{enumerate}

\item El Decreto 135/2016, de 10 de octubre, por la que se aprueba el Reglamento
Orgánico de la Consejería de Educación y Universidades. (\enquote{BOC} 203 de 19
de octubre), establece en su artículo 17.2.c) que entre sus funciones les 
corresponde a las Direcciones Territoriales de Educación:  \enquote{Conceder 
licencias y permisos al personal docente no universitario en relación con las 
competencias atribuidas}.

Dado que \VAR{dondona} \VAR{nombre} se jubila con anterioridad a la finalización
 del periodo ordinario de vacaciones, y tiene derecho al disfrute de las 
 vacaciones durante cada curso escolar y en uso de las  competencias atribuidas 
 en el referido Decreto 135/2016, de 10 de octubre,

\end{enumerate}

\subsection*{RESUELVO}

\begin{enumerate}

\item Conceder el disfrute de las vacaciones a \VAR{dondona} \VAR{nombre}, con 
DNI \VAR{dni}, el periodo comprendido \VAR{periodo}.

\item Notificar la presente resolución \VAR{al_interesadoa}.

\end{enumerate}

\recurso